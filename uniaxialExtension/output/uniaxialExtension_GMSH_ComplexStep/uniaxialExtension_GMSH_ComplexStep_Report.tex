\documentclass[a4paper,10pt]{article} 
\usepackage[a4paper,margin=20mm]{geometry} 
\usepackage{longtable} 
\usepackage{float} 
\usepackage{adjustbox} 
\usepackage{graphicx} 
\usepackage{color} 
\usepackage{booktabs} 
\aboverulesep=0ex 
\belowrulesep=0ex 
\usepackage{array} 
\newcolumntype{?}{!{\vrule width 1pt}}
\usepackage{fancyhdr} 
\pagestyle{fancy} 
\fancyhf{} 
\rhead{Problem: uniaxialExtensionGMSHComplexStep } 
\lfoot{Date: \today} 
\rfoot{Page \thepage} 
\renewcommand{\footrulewidth}{1pt} 
\renewcommand{\headrulewidth}{1.5pt} 
\setlength{\parindent}{0pt} 
\usepackage[T1]{fontenc} 
\usepackage{libertine} 
\usepackage{arydshln} 
\definecolor{miblue}{rgb}{0,0.1,0.38} 
\usepackage{titlesec} 
\titleformat{\section}{\normalfont\Large\color{miblue}\bfseries}{\color{miblue}\sectionmark\thesection}{0.5em}{}[{\color{miblue}\titlerule[0.5pt]}] 

\begin{document} 
This is an ONSAS automatically-generated report with part of the results obtained after the analysis. The user can access other magnitudes and results through the GNU-Octave/MATLAB console. The code is provided AS IS \textbf{WITHOUT WARRANTY of any kind}, express or implied.

\section{Analysis results}

\begin{longtable}{cccccc} 
$\#t$ & $t$ & its & $\| RHS \|$ & $\| \Delta u \|$ & flagExit \\ \hline 
 \endhead 
   1 &  0.00e+00 &    0 &           &           &   \\ 
 \hdashline 
     &           &    1 &  9.31e-02 &  3.77e+00 &      \\ 
     &           &    2 &  8.02e-03 &  9.88e-01 &      \\ 
     &           &    3 &  1.13e-04 &  1.25e-01 &      \\ 
     &           &    4 &  2.05e-08 &  1.66e-03 &      \\ 
     &           &    5 &  7.89e-16 &  3.14e-07 &      \\ 
   2 &  1.25e-01 &    5 &           &           & forces  \\ 
 \hdashline 
     &           &    1 &  3.09e-02 &  2.04e+00 &      \\ 
     &           &    2 &  7.28e-04 &  2.85e-01 &      \\ 
     &           &    3 &  6.12e-07 &  8.73e-03 &      \\ 
     &           &    4 &  3.72e-13 &  6.68e-06 &      \\ 
   3 &  2.50e-01 &    4 &           &           & forces  \\ 
 \hdashline 
     &           &    1 &  1.90e-02 &  1.53e+00 &      \\ 
     &           &    2 &  2.04e-04 &  1.43e-01 &      \\ 
     &           &    3 &  3.54e-08 &  2.03e-03 &      \\ 
     &           &    4 &  1.19e-15 &  3.20e-07 &      \\ 
   4 &  3.75e-01 &    4 &           &           & forces  \\ 
 \hdashline 
     &           &    1 &  1.41e-02 &  1.28e+00 &      \\ 
     &           &    2 &  8.71e-05 &  8.89e-02 &      \\ 
     &           &    3 &  4.56e-09 &  6.95e-04 &      \\ 
   5 &  5.00e-01 &    3 &           &           & forces  \\ 
 \hdashline 
     &           &    1 &  1.15e-02 &  1.13e+00 &      \\ 
     &           &    2 &  4.91e-05 &  6.27e-02 &      \\ 
     &           &    3 &  9.23e-10 &  2.68e-04 &      \\ 
   6 &  6.25e-01 &    3 &           &           & forces  \\ 
 \hdashline 
     &           &    1 &  1.00e-02 &  1.04e+00 &      \\ 
     &           &    2 &  4.08e-05 &  5.11e-02 &      \\ 
     &           &    3 &  4.74e-09 &  4.40e-04 &      \\ 
   7 &  7.50e-01 &    3 &           &           & forces  \\ 
 \hdashline 
     &           &    1 &  9.32e-03 &  1.00e+00 &      \\ 
     &           &    2 &  5.89e-05 &  5.44e-02 &      \\ 
     &           &    3 &  4.17e-08 &  1.51e-03 &      \\ 
     &           &    4 &  2.66e-14 &  1.22e-06 &      \\ 
   8 &  8.75e-01 &    4 &           &           & forces  \\ 
 \hdashline 
     &           &    1 &  9.41e-03 &  1.01e+00 &      \\ 
     &           &    2 &  1.39e-04 &  8.68e-02 &      \\ 
     &           &    3 &  7.07e-07 &  7.09e-03 &      \\ 
     &           &    4 &  1.74e-11 &  3.52e-05 &      \\ 
   9 &  1.00e+00 &    4 &           &           & forces  \\ 
 \hdashline 
 
\end{longtable}

\end{document}