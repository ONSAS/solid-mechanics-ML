\documentclass[a4paper,10pt]{article} 
\usepackage[a4paper,margin=20mm]{geometry} 
\usepackage{longtable} 
\usepackage{float} 
\usepackage{adjustbox} 
\usepackage{graphicx} 
\usepackage{color} 
\usepackage{booktabs} 
\aboverulesep=0ex 
\belowrulesep=0ex 
\usepackage{array} 
\newcolumntype{?}{!{\vrule width 1pt}}
\usepackage{fancyhdr} 
\pagestyle{fancy} 
\fancyhf{} 
\rhead{Problem: uniaxialExtensionNHC } 
\lfoot{Date: \today} 
\rfoot{Page \thepage} 
\renewcommand{\footrulewidth}{1pt} 
\renewcommand{\headrulewidth}{1.5pt} 
\setlength{\parindent}{0pt} 
\usepackage[T1]{fontenc} 
\usepackage{libertine} 
\usepackage{arydshln} 
\definecolor{miblue}{rgb}{0,0.1,0.38} 
\usepackage{titlesec} 
\titleformat{\section}{\normalfont\Large\color{miblue}\bfseries}{\color{miblue}\sectionmark\thesection}{0.5em}{}[{\color{miblue}\titlerule[0.5pt]}] 

\begin{document} 
This is an ONSAS automatically-generated report with part of the results obtained after the analysis. The user can access other magnitudes and results through the GNU-Octave/MATLAB console. The code is provided AS IS \textbf{WITHOUT WARRANTY of any kind}, express or implied.

\section{Analysis results}

\begin{longtable}{cccccc} 
$\#t$ & $t$ & its & $\| RHS \|$ & $\| \Delta u \|$ & flagExit \\ \hline 
 \endhead 
   1 &  0.00e+00 &    0 &           &           &   \\ 
 \hdashline 
     &           &    1 &  5.33e-02 &  3.68e+00 &      \\ 
     &           &    2 &  2.15e-03 &  1.07e+00 &      \\ 
     &           &    3 &  2.35e-05 &  7.54e-02 &      \\ 
     &           &    4 &  3.19e-10 &  8.10e-05 &      \\ 
   2 &  1.25e-01 &    4 &           &           & forces  \\ 
 \hdashline 
     &           &    1 &  1.06e-01 &  6.03e+00 &      \\ 
     &           &    2 &  8.10e-03 &  1.06e+00 &      \\ 
     &           &    3 &  9.59e-06 &  3.24e-02 &      \\ 
     &           &    4 &  4.99e-11 &  3.09e-04 &      \\ 
   3 &  2.50e-01 &    4 &           &           & forces  \\ 
 \hdashline 
     &           &    1 &  1.27e-01 &  7.97e+00 &      \\ 
     &           &    2 &  7.13e-03 &  5.60e-01 &      \\ 
     &           &    3 &  1.63e-05 &  1.45e-02 &      \\ 
     &           &    4 &  1.01e-10 &  6.64e-05 &      \\ 
   4 &  3.75e-01 &    4 &           &           & forces  \\ 
 \hdashline 
     &           &    1 &  1.08e-01 &  8.88e+00 &      \\ 
     &           &    2 &  3.56e-03 &  2.40e-01 &      \\ 
     &           &    3 &  3.72e-06 &  9.89e-03 &      \\ 
     &           &    4 &  3.93e-12 &  8.60e-06 &      \\ 
   5 &  5.00e-01 &    4 &           &           & forces  \\ 
 \hdashline 
     &           &    1 &  8.40e-02 &  9.24e+00 &      \\ 
     &           &    2 &  1.72e-03 &  1.11e-01 &      \\ 
     &           &    3 &  7.51e-07 &  4.49e-03 &      \\ 
     &           &    4 &  1.37e-13 &  1.41e-06 &      \\ 
   6 &  6.25e-01 &    4 &           &           & forces  \\ 
 \hdashline 
     &           &    1 &  6.56e-02 &  9.40e+00 &      \\ 
     &           &    2 &  8.98e-04 &  5.91e-02 &      \\ 
     &           &    3 &  1.81e-07 &  2.03e-03 &      \\ 
     &           &    4 &  6.99e-15 &  2.96e-07 &      \\ 
   7 &  7.50e-01 &    4 &           &           & forces  \\ 
 \hdashline 
     &           &    1 &  5.25e-02 &  9.47e+00 &      \\ 
     &           &    2 &  5.13e-04 &  3.54e-02 &      \\ 
     &           &    3 &  5.27e-08 &  9.80e-04 &      \\ 
     &           &    4 &  1.97e-15 &  7.66e-08 &      \\ 
   8 &  8.75e-01 &    4 &           &           & forces  \\ 
 \hdashline 
     &           &    1 &  4.29e-02 &  9.51e+00 &      \\ 
     &           &    2 &  3.14e-04 &  2.32e-02 &      \\ 
     &           &    3 &  1.80e-08 &  5.11e-04 &      \\ 
     &           &    4 &  2.28e-15 &  2.33e-08 &      \\ 
   9 &  1.00e+00 &    4 &           &           & forces  \\ 
 \hdashline 
 
\end{longtable}

\end{document}