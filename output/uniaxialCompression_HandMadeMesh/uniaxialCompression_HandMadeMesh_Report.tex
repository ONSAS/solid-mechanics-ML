\documentclass[a4paper,10pt]{article} 
\usepackage[a4paper,margin=20mm]{geometry} 
\usepackage{longtable} 
\usepackage{float} 
\usepackage{adjustbox} 
\usepackage{graphicx} 
\usepackage{color} 
\usepackage{booktabs} 
\aboverulesep=0ex 
\belowrulesep=0ex 
\usepackage{array} 
\newcolumntype{?}{!{\vrule width 1pt}}
\usepackage{fancyhdr} 
\pagestyle{fancy} 
\fancyhf{} 
\rhead{Problem: uniaxialCompressionHandMadeMesh } 
\lfoot{Date: \today} 
\rfoot{Page \thepage} 
\renewcommand{\footrulewidth}{1pt} 
\renewcommand{\headrulewidth}{1.5pt} 
\setlength{\parindent}{0pt} 
\usepackage[T1]{fontenc} 
\usepackage{libertine} 
\usepackage{arydshln} 
\definecolor{miblue}{rgb}{0,0.1,0.38} 
\usepackage{titlesec} 
\titleformat{\section}{\normalfont\Large\color{miblue}\bfseries}{\color{miblue}\sectionmark\thesection}{0.5em}{}[{\color{miblue}\titlerule[0.5pt]}] 

\begin{document} 
This is an ONSAS automatically-generated report with part of the results obtained after the analysis. The user can access other magnitudes and results through the GNU-Octave/MATLAB console. The code is provided AS IS \textbf{WITHOUT WARRANTY of any kind}, express or implied.

\section{Analysis results}

\begin{longtable}{cccccc} 
$\#t$ & $t$ & its & $\| RHS \|$ & $\| \Delta u \|$ & flagExit \\ \hline 
 \endhead 
   1 &  0.00e+00 &    0 &           &           &   \\ 
 \hdashline 
     &           &    1 &  1.95e-02 &  4.42e-01 &      \\ 
     &           &    2 &  4.83e-05 &  2.37e-02 &      \\ 
     &           &    3 &  1.49e-09 &  9.60e-05 &      \\ 
   2 &  1.00e-01 &    3 &           &           & forces  \\ 
 \hdashline 
     &           &    1 &  1.63e-02 &  3.95e-01 &      \\ 
     &           &    2 &  3.96e-05 &  2.11e-02 &      \\ 
     &           &    3 &  1.00e-09 &  8.04e-05 &      \\ 
   3 &  2.00e-01 &    3 &           &           & forces  \\ 
 \hdashline 
     &           &    1 &  1.37e-02 &  3.54e-01 &      \\ 
     &           &    2 &  3.34e-05 &  1.86e-02 &      \\ 
     &           &    3 &  6.49e-10 &  6.55e-05 &      \\ 
   4 &  3.00e-01 &    3 &           &           & forces  \\ 
 \hdashline 
     &           &    1 &  1.18e-02 &  3.17e-01 &      \\ 
     &           &    2 &  2.88e-05 &  1.63e-02 &      \\ 
     &           &    3 &  4.12e-10 &  5.25e-05 &      \\ 
   5 &  4.00e-01 &    3 &           &           & forces  \\ 
 \hdashline 
     &           &    1 &  1.03e-02 &  2.85e-01 &      \\ 
     &           &    2 &  2.49e-05 &  1.42e-02 &      \\ 
     &           &    3 &  2.63e-10 &  4.15e-05 &      \\ 
   6 &  5.00e-01 &    3 &           &           & forces  \\ 
 \hdashline 
     &           &    1 &  9.18e-03 &  2.57e-01 &      \\ 
     &           &    2 &  2.16e-05 &  1.23e-02 &      \\ 
     &           &    3 &  1.70e-10 &  3.27e-05 &      \\ 
   7 &  6.00e-01 &    3 &           &           & forces  \\ 
 \hdashline 
     &           &    1 &  8.34e-03 &  2.33e-01 &      \\ 
     &           &    2 &  1.88e-05 &  1.08e-02 &      \\ 
     &           &    3 &  1.13e-10 &  2.57e-05 &      \\ 
   8 &  7.00e-01 &    3 &           &           & forces  \\ 
 \hdashline 
     &           &    1 &  7.70e-03 &  2.12e-01 &      \\ 
     &           &    2 &  1.63e-05 &  9.38e-03 &      \\ 
     &           &    3 &  7.65e-11 &  2.02e-05 &      \\ 
   9 &  8.00e-01 &    3 &           &           & forces  \\ 
 \hdashline 
     &           &    1 &  7.20e-03 &  1.94e-01 &      \\ 
     &           &    2 &  1.41e-05 &  8.21e-03 &      \\ 
     &           &    3 &  5.31e-11 &  1.60e-05 &      \\ 
  10 &  9.00e-01 &    3 &           &           & forces  \\ 
 \hdashline 
     &           &    1 &  6.80e-03 &  1.78e-01 &      \\ 
     &           &    2 &  1.23e-05 &  7.20e-03 &      \\ 
     &           &    3 &  3.76e-11 &  1.27e-05 &      \\ 
  11 &  1.00e+00 &    3 &           &           & forces  \\ 
 \hdashline 
 
\end{longtable}

\end{document}