\documentclass[a4paper,10pt]{article} 
\usepackage[a4paper,margin=20mm]{geometry} 
\usepackage{longtable} 
\usepackage{float} 
\usepackage{adjustbox} 
\usepackage{graphicx} 
\usepackage{color} 
\usepackage{booktabs} 
\aboverulesep=0ex 
\belowrulesep=0ex 
\usepackage{array} 
\newcolumntype{?}{!{\vrule width 1pt}}
\usepackage{fancyhdr} 
\pagestyle{fancy} 
\fancyhf{} 
\rhead{Problem: uniaxialExtensionHandMadeMesh } 
\lfoot{Date: \today} 
\rfoot{Page \thepage} 
\renewcommand{\footrulewidth}{1pt} 
\renewcommand{\headrulewidth}{1.5pt} 
\setlength{\parindent}{0pt} 
\usepackage[T1]{fontenc} 
\usepackage{libertine} 
\usepackage{arydshln} 
\definecolor{miblue}{rgb}{0,0.1,0.38} 
\usepackage{titlesec} 
\titleformat{\section}{\normalfont\Large\color{miblue}\bfseries}{\color{miblue}\sectionmark\thesection}{0.5em}{}[{\color{miblue}\titlerule[0.5pt]}] 

\begin{document} 
This is an ONSAS automatically-generated report with part of the results obtained after the analysis. The user can access other magnitudes and results through the GNU-Octave/MATLAB console. The code is provided AS IS \textbf{WITHOUT WARRANTY of any kind}, express or implied.

\section{Analysis results}

\begin{longtable}{cccccc} 
$\#t$ & $t$ & its & $\| RHS \|$ & $\| \Delta u \|$ & flagExit \\ \hline 
 \endhead 
   1 &  0.00e+00 &    0 &           &           &   \\ 
 \hdashline 
     &           &    1 &  1.64e-01 &  1.53e+00 &      \\ 
     &           &    2 &  1.39e-02 &  4.02e-01 &      \\ 
     &           &    3 &  1.98e-04 &  5.07e-02 &      \\ 
     &           &    4 &  3.56e-08 &  6.76e-04 &      \\ 
     &           &    5 &  1.34e-15 &  1.28e-07 &      \\ 
   2 &  1.25e-01 &    5 &           &           & forces  \\ 
 \hdashline 
     &           &    1 &  5.43e-02 &  8.27e-01 &      \\ 
     &           &    2 &  1.25e-03 &  1.16e-01 &      \\ 
     &           &    3 &  1.06e-06 &  3.55e-03 &      \\ 
     &           &    4 &  6.42e-13 &  2.72e-06 &      \\ 
   3 &  2.50e-01 &    4 &           &           & forces  \\ 
 \hdashline 
     &           &    1 &  3.33e-02 &  6.21e-01 &      \\ 
     &           &    2 &  3.52e-04 &  5.84e-02 &      \\ 
     &           &    3 &  6.13e-08 &  8.23e-04 &      \\ 
     &           &    4 &  1.60e-15 &  1.30e-07 &      \\ 
   4 &  3.75e-01 &    4 &           &           & forces  \\ 
 \hdashline 
     &           &    1 &  2.46e-02 &  5.18e-01 &      \\ 
     &           &    2 &  1.50e-04 &  3.62e-02 &      \\ 
     &           &    3 &  7.85e-09 &  2.82e-04 &      \\ 
   5 &  5.00e-01 &    3 &           &           & forces  \\ 
 \hdashline 
     &           &    1 &  2.01e-02 &  4.58e-01 &      \\ 
     &           &    2 &  8.60e-05 &  2.55e-02 &      \\ 
     &           &    3 &  1.63e-09 &  1.09e-04 &      \\ 
   6 &  6.25e-01 &    3 &           &           & forces  \\ 
 \hdashline 
     &           &    1 &  1.75e-02 &  4.22e-01 &      \\ 
     &           &    2 &  7.48e-05 &  2.07e-02 &      \\ 
     &           &    3 &  9.34e-09 &  1.76e-04 &      \\ 
   7 &  7.50e-01 &    3 &           &           & forces  \\ 
 \hdashline 
     &           &    1 &  1.63e-02 &  4.05e-01 &      \\ 
     &           &    2 &  1.13e-04 &  2.19e-02 &      \\ 
     &           &    3 &  8.16e-08 &  6.03e-04 &      \\ 
     &           &    4 &  5.19e-14 &  4.89e-07 &      \\ 
   8 &  8.75e-01 &    4 &           &           & forces  \\ 
 \hdashline 
     &           &    1 &  1.64e-02 &  4.08e-01 &      \\ 
     &           &    2 &  2.69e-04 &  3.47e-02 &      \\ 
     &           &    3 &  1.36e-06 &  2.83e-03 &      \\ 
     &           &    4 &  3.36e-11 &  1.41e-05 &      \\ 
   9 &  1.00e+00 &    4 &           &           & forces  \\ 
 \hdashline 
 
\end{longtable}

\end{document}