\documentclass[a4paper,10pt]{article} 
\usepackage[a4paper,margin=20mm]{geometry} 
\usepackage{longtable} 
\usepackage{float} 
\usepackage{adjustbox} 
\usepackage{graphicx} 
\usepackage{color} 
\usepackage{booktabs} 
\aboverulesep=0ex 
\belowrulesep=0ex 
\usepackage{array} 
\newcolumntype{?}{!{\vrule width 1pt}}
\usepackage{fancyhdr} 
\pagestyle{fancy} 
\fancyhf{} 
\rhead{Problem: cantileversolid } 
\lfoot{Date: \today} 
\rfoot{Page \thepage} 
\renewcommand{\footrulewidth}{1pt} 
\renewcommand{\headrulewidth}{1.5pt} 
\setlength{\parindent}{0pt} 
\usepackage[T1]{fontenc} 
\usepackage{libertine} 
\usepackage{arydshln} 
\definecolor{miblue}{rgb}{0,0.1,0.38} 
\usepackage{titlesec} 
\titleformat{\section}{\normalfont\Large\color{miblue}\bfseries}{\color{miblue}\sectionmark\thesection}{0.5em}{}[{\color{miblue}\titlerule[0.5pt]}] 

\begin{document} 
This is an ONSAS automatically-generated report with part of the results obtained after the analysis. The user can access other magnitudes and results through the GNU-Octave/MATLAB console. The code is provided AS IS \textbf{WITHOUT WARRANTY of any kind}, express or implied.

\section{Analysis results}

\begin{longtable}{cccccc} 
$\#t$ & $t$ & its & $\| RHS \|$ & $\| \Delta u \|$ & flagExit \\ \hline 
 \endhead 
   1 &  0.00e+00 &    0 &           &           &   \\ 
 \hdashline 
     &           &    1 &  1.09e-03 &  1.66e-02 &      \\ 
     &           &    2 &  5.32e-07 &  6.68e-05 &      \\ 
     &           &    3 &  1.74e-09 &  9.37e-08 &      \\ 
     &           &    4 &  2.49e-12 &  2.00e-10 &      \\ 
   2 &  5.00e-02 &    4 &           &           & forces  \\ 
 \hdashline 
     &           &    1 &  1.53e-05 &  5.04e-04 &      \\ 
     &           &    2 &  3.17e-08 &  2.20e-06 &      \\ 
     &           &    3 &  3.69e-11 &  2.57e-09 &      \\ 
   3 &  1.00e-01 &    3 &           &           & forces  \\ 
 \hdashline 
     &           &    1 &  1.57e-05 &  5.01e-04 &      \\ 
     &           &    2 &  3.14e-08 &  2.12e-06 &      \\ 
     &           &    3 &  4.04e-11 &  3.04e-09 &      \\ 
   4 &  1.50e-01 &    3 &           &           & forces  \\ 
 \hdashline 
     &           &    1 &  1.62e-05 &  4.98e-04 &      \\ 
     &           &    2 &  3.17e-08 &  2.04e-06 &      \\ 
     &           &    3 &  4.55e-11 &  3.60e-09 &      \\ 
   5 &  2.00e-01 &    3 &           &           & forces  \\ 
 \hdashline 
     &           &    1 &  1.69e-05 &  4.95e-04 &      \\ 
     &           &    2 &  3.24e-08 &  1.98e-06 &      \\ 
     &           &    3 &  5.22e-11 &  4.24e-09 &      \\ 
   6 &  2.50e-01 &    3 &           &           & forces  \\ 
 \hdashline 
     &           &    1 &  1.77e-05 &  4.92e-04 &      \\ 
     &           &    2 &  3.34e-08 &  1.92e-06 &      \\ 
     &           &    3 &  6.08e-11 &  4.95e-09 &      \\ 
   7 &  3.00e-01 &    3 &           &           & forces  \\ 
 \hdashline 
     &           &    1 &  1.85e-05 &  4.90e-04 &      \\ 
     &           &    2 &  3.48e-08 &  1.87e-06 &      \\ 
     &           &    3 &  7.17e-11 &  5.74e-09 &      \\ 
   8 &  3.50e-01 &    3 &           &           & forces  \\ 
 \hdashline 
     &           &    1 &  1.94e-05 &  4.87e-04 &      \\ 
     &           &    2 &  3.65e-08 &  1.82e-06 &      \\ 
     &           &    3 &  8.59e-11 &  6.59e-09 &      \\ 
   9 &  4.00e-01 &    3 &           &           & forces  \\ 
 \hdashline 
     &           &    1 &  2.03e-05 &  4.84e-04 &      \\ 
     &           &    2 &  3.86e-08 &  1.78e-06 &      \\ 
     &           &    3 &  1.04e-10 &  7.51e-09 &      \\ 
  10 &  4.50e-01 &    3 &           &           & forces  \\ 
 \hdashline 
     &           &    1 &  2.13e-05 &  4.82e-04 &      \\ 
     &           &    2 &  4.10e-08 &  1.75e-06 &      \\ 
     &           &    3 &  1.27e-10 &  8.50e-09 &      \\ 
  11 &  5.00e-01 &    3 &           &           & forces  \\ 
 \hdashline 
     &           &    1 &  2.24e-05 &  4.79e-04 &      \\ 
     &           &    2 &  4.40e-08 &  1.73e-06 &      \\ 
     &           &    3 &  1.54e-10 &  9.55e-09 &      \\ 
  12 &  5.50e-01 &    3 &           &           & forces  \\ 
 \hdashline 
     &           &    1 &  2.34e-05 &  4.77e-04 &      \\ 
     &           &    2 &  4.74e-08 &  1.71e-06 &      \\ 
     &           &    3 &  1.88e-10 &  1.07e-08 &      \\ 
  13 &  6.00e-01 &    3 &           &           & forces  \\ 
 \hdashline 
     &           &    1 &  2.44e-05 &  4.74e-04 &      \\ 
     &           &    2 &  5.13e-08 &  1.70e-06 &      \\ 
     &           &    3 &  2.27e-10 &  1.19e-08 &      \\ 
  14 &  6.50e-01 &    3 &           &           & forces  \\ 
 \hdashline 
     &           &    1 &  2.55e-05 &  4.72e-04 &      \\ 
     &           &    2 &  5.58e-08 &  1.69e-06 &      \\ 
     &           &    3 &  2.72e-10 &  1.31e-08 &      \\ 
  15 &  7.00e-01 &    3 &           &           & forces  \\ 
 \hdashline 
     &           &    1 &  2.66e-05 &  4.70e-04 &      \\ 
     &           &    2 &  6.08e-08 &  1.69e-06 &      \\ 
     &           &    3 &  3.24e-10 &  1.44e-08 &      \\ 
  16 &  7.50e-01 &    3 &           &           & forces  \\ 
 \hdashline 
     &           &    1 &  2.77e-05 &  4.68e-04 &      \\ 
     &           &    2 &  6.64e-08 &  1.70e-06 &      \\ 
     &           &    3 &  3.83e-10 &  1.58e-08 &      \\ 
  17 &  8.00e-01 &    3 &           &           & forces  \\ 
 \hdashline 
     &           &    1 &  2.87e-05 &  4.66e-04 &      \\ 
     &           &    2 &  7.25e-08 &  1.71e-06 &      \\ 
     &           &    3 &  4.48e-10 &  1.72e-08 &      \\ 
  18 &  8.50e-01 &    3 &           &           & forces  \\ 
 \hdashline 
     &           &    1 &  2.98e-05 &  4.64e-04 &      \\ 
     &           &    2 &  7.91e-08 &  1.72e-06 &      \\ 
     &           &    3 &  5.21e-10 &  1.87e-08 &      \\ 
  19 &  9.00e-01 &    3 &           &           & displac  \\ 
 \hdashline 
     &           &    1 &  3.09e-05 &  4.62e-04 &      \\ 
     &           &    2 &  8.63e-08 &  1.74e-06 &      \\ 
     &           &    3 &  6.00e-10 &  2.02e-08 &      \\ 
     &           &    4 &  1.44e-12 &  3.76e-11 &      \\ 
  20 &  9.50e-01 &    4 &           &           & forces  \\ 
 \hdashline 
     &           &    1 &  3.20e-05 &  4.60e-04 &      \\ 
     &           &    2 &  9.39e-08 &  1.77e-06 &      \\ 
     &           &    3 &  6.87e-10 &  2.18e-08 &      \\ 
     &           &    4 &  1.74e-12 &  4.32e-11 &      \\ 
  21 &  1.00e+00 &    4 &           &           & forces  \\ 
 \hdashline 
 
\end{longtable}

\end{document}